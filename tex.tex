\documentclass{article}

\usepackage{biblatex}
\addbibresource{bib.bib}

\usepackage[utf8]{inputenc}
\usepackage{amsmath}
\usepackage{amssymb}
\usepackage{amsthm}
\usepackage{amsfonts}
\usepackage{stmaryrd}
\usepackage{mathrsfs}
\newtheorem{theorem}{Theorem}[section]
\newtheorem{corollary}{Corollary}[theorem]
\newtheorem{lemma}[theorem]{Lemma}
\newtheorem{proposition}[theorem]{Proposition}
\theoremstyle{definition}
\newtheorem{definition}{Definition}[section]
\newtheorem{example}{Example}[section]
\newtheorem{exercise}{Exercise}[section]
\newtheorem{solution}{Solution}[exercise]
\usepackage{xcolor}
\usepackage{graphicx}
\usepackage{tikz}
\usepackage{tikz-cd}
\usepackage{tkz-euclide}
\usepackage{hyperref}
\usepackage{bm}
\usepackage{quiver} 
%\usepackage[a5paper, total={108mm, 160mm}]{geometry}

\pagecolor[HTML]{ECE1CB} %beige

\color[HTML]{756244} %brown 

\newtheorem{case}{Case}
\newcommand{\dd}[2]{\frac{\partial #1}{\partial #2}}
\newcommand{\oring}{\mathcal{O}}
\newcommand{\id}{\operatorname{id}}
\newcommand{\h}[1]{\widehat{#1}}
\newcommand{\height}{\operatorname{ht}}
\newcommand{\supp}{\operatorname{Supp}}
\newcommand{\im}{\operatorname{im}}
\newcommand{\coim}{\operatorname{coim}}

\newcommand{\ext}[1][1]{\operatorname{Ext}^{#1}}
\newcommand{\C}{\mathbb{C}}
\newcommand{\R}{\mathbb{R}}
\newcommand{\Q}{\mathbb{Q}}
\newcommand{\Z}{\mathbb{Z}}
\newcommand{\N}{\mathbb{N}}
\newcommand{\Pp}{\mathbb{P}}
\newcommand{\p}{\mathfrak{p}}
\newcommand{\q}{\mathfrak{q}}
\newcommand{\m}{\mathfrak{m}}
\newcommand{\del}{\partial}
\newcommand{\Gen}[1]{\langle #1 \rangle}
\newcommand{\spec}{\operatorname{spec}}
\newcommand{\Lim}[1]{\raisebox{0.5ex}{\scalebox{0.8}{$\displaystyle \lim_{#1}\;$}}}
\newcommand{\IN}{\operatorname{in}_{\prec}}
\newcommand{\ann}{\operatorname{ann}}
\newcommand{\Ass}{\operatorname{Ass}}
\newcommand{\coker}{\operatorname{coker}}
\newcommand{\pic}{\mathcal{P}ic}
\newcommand{\res}{\operatorname{res}}
\newcommand{\A}[2]{\mathbb{A}_#1^{#2}}
\newcommand{\Pj}[2]{\mathbb{P}_#1^{\#2}}
\newcommand{\cK}{\mathcal{K}}
\newcommand{\cM}{\mathcal{M}}
\newcommand{\cF}{\mathcal{F}}
\newcommand{\br}{\vspace{5mm}}

\newcommand{\V}{\mathbb{V}}
\newcommand{\I}{\mathbb{I}}

\newcommand{\cC}{\mathcal{C}}
\newcommand{\cD}{\mathcal{D}}
\newcommand{\cT}{\mathcal{T}}

\newcommand{\lil}{{\lambda \in \Lambda}}
\newcommand{\cL}{\mathcal{L}}
\newcommand{\cc}{\mathcal}
\newcommand{\rightrightrightarrows}{\substack{\rightarrow\\[-1em] \rightarrow \\[-1em] \rightarrow}}



\newcommand{\nul}{Nullstellensatz }

% \usepackage{ccfonts}
% \usepackage[T1]{fontenc}
% \usepackage{titlesec}

\usepackage[capitalise]{cleveref}

\title{Exercises on Derived Categories}
\author{kyjthompson }
\date{November 2025}

\begin{document}

\maketitle

I need to do exercises so I found this \href{https://homepages.math.uic.edu/~bshipley/krause.chicago_ex.pdf}{Exercises on derived categories, resolutions, and Brown
representability} to I suppose work through 

\section{}
\subsection{}
\begin{exercise}
Let $\mathcal{A}$ be an abelian category. Show that $\mathbf{K}(\mathcal A)$ and $\mathbf D (\mathcal A)$ are additive categories, and that the canonical functor $\mathbf{K} (\mathcal A) \to \mathbf D (\mathcal A)$ is additive
\end{exercise}
\begin{solution}
Clearly the hom sets for the category $\mathbf K (\mathcal A)$ admits an abelian group structure by just adding in each component in $K(\mathcal{A})$ and then noting that \[H=\{f \vert f \simeq 0\}=\{f \vert \exists \phi, f = d\phi + \phi d\}\] is a subgroup as $d\phi + \phi d - (d\mu + \mu d)=d(\phi-\mu)+(\phi-\mu)d$ and $0 \in H$ so the structure decends naturally to the quotient. Since $\{q-iso\}$ form a localising class in $\mathbf K (\mathcal A)$ we can consider morphisms in $\mathbf{D} (\mathcal  A)$ as spans $X \rightarrow Z \leftarrow Y$ which have an abelian group structure since for two morphisms $(s \colon X \rightarrow Z \leftarrow Y \colon f )+ (s'\colon X \rightarrow  Z' \leftarrow Y \colon f')$ we can define their sum as follows, first we use the extensionality to construct some $W$ over both $Z, Z'$ so that the square 
\[\begin{tikzcd}
W \arrow[d,dashed, "a'"'] \arrow[r, dashed, "a"] & Z \arrow[d] \\
Z' \arrow[r]                      & X          
\end{tikzcd}\]
commutes and has all morphisms quasi iso, we can then construct the span 
\[\begin{tikzcd}
W \arrow[r, "f \circ a + f' \circ a'"] \arrow[d] & Y \\
X                                                &  
\end{tikzcd}\]
as the sum of our two morphisms. Since sadly this choice of $W$ isn't canonical, we must check that for any other choice of $W$ we get an equivalent span. This is surprisingly annoying. We can do so by taking some other common roof say $W'$ with projections onto the $Z$'s $b,b'$ we can find an object $W^\circ$ from the following extensions
\[\begin{tikzcd}
W^\Delta \arrow[d, dashed] \arrow[r, dashed] & W' \arrow[d] & W^\square \arrow[d, dashed] \arrow[r, dashed] & W' \arrow[d] & W^\circ \arrow[d, dashed] \arrow[r, dashed] & W^\square \arrow[d] \\
W \arrow[r]                                  & Z            & W \arrow[r]                                   & Z'           & W^\Delta \arrow[r]                          & X                  
\end{tikzcd}\]
This however lacks canonical morphisms to $W,W'$ as we have two natural routes, this can be fixed by noting that the maps $W^\circ \rightrightarrows W$ are equalised once we get to $X$ and so since $W \to X$ is a quasi iso they must be equalised from some quasi iso $V_* \to W^\circ$, doing the same for $W'$ we get two maps to $W^\circ$ so applying the extensionality one more time gives us some $V$ with quasi isomorphisms $\mu\colon V \to W, \mu' \colon V \to W'$ satisfying $a\mu=b\mu', a'\mu=b'\mu'$ and is hense a common roof for our two sums.


We clearly have identity $id\colon X \rightarrow X \leftarrow Y \colon 0$ since for anything else $Z$ is a common roof so we just end up with $f+0=f:Z \to Y$ and $s:Z \to X$ so nothing chages. For any $(s \colon X \rightarrow Z \leftarrow Y \colon f )$ we have inverse $(s \colon X \rightarrow Z \leftarrow Y \colon -f )$ as in that case we can choose $W = Z$ leaving the triangle 
\[\begin{tikzcd}
Z \arrow[r, "0"] \arrow[d] & Y \\
X                          &  
\end{tikzcd}\]
Which is clearly equivalent to our identity element by the roof 
\[
\begin{tikzcd}
                                                 & Z \arrow[ld] \arrow[d, Rightarrow,no head,] \\
X \arrow[d, Rightarrow,no head,] \arrow[rd, "0"] & Z \arrow[d, "0"] \arrow[ld]                 \\
X                                                & Y                                          
\end{tikzcd}
\]
So we've shown that $\mathbf{K}(\mathcal{A}), \mathbf{D}(\mathcal{A})$ are preadditive, we now need to show that they have zero objects and products. We obviously have a zero objects in $K(\mathcal{A})$ given by the zero complex, this then decends to a zero object in $\mathbf{K}(\mathcal{A})$ as the homotopy relation can only make the hom sets smaller, then in the derived category we have that the zero morphism is a quasi isomorphism from the zero complex to itself and hense this is a zero object in $\mathbf{D}(\mathcal{A})$ too.

For finite products we just need to check for binary products. We can find the product of two complexes by either doing $(A \oplus B)_i = A_i \oplus B_i$ or $(A \oplus B)_i = \bigoplus_{m+n=i}A_m \oplus B_n$ and honestly I'm not sure which. Ok there's a problem with the second one which is good for me. The problem is we would need potentially infinite products in $\mathcal A$ which may not exist. Instead we will try the first one, which honestly I much prefer to work with anyway. Given two complexes $A,B$ we can construct the product complex $A \oplus B$ having 
\[
(A \oplus B)^i = A^i \oplus B^i \quad d^i = \begin{pmatrix}d_A & 0 \\ 0 & d_b\end{pmatrix}
\]
We can see that this is a product in $K(\mathcal{A})$ by zooming in to degree $0$ and noting that a map $Z \to A \oplus B$ we get the diagram
\[
\begin{tikzcd}
\vdots                                 & \vdots                   \\
Z^1 \arrow[u] \arrow[r, "{(f^1,g^1)}"] & A_1 \oplus B_1 \arrow[u] \\
Z^0 \arrow[u] \arrow[r, "{(f^0,g^0)}"] & A_0 \oplus B_0 \arrow[u] \\
\vdots \arrow[u]                       & \vdots \arrow[u]        
\end{tikzcd}
\]
Which clearly commutes iff $df=fd$ and $dg=gd$ ie that we have a map to $A$ and one to $B$, this is precicely the universal property of the product. This structure then descends to $\mathbf K (\mathcal{A})$ as for the same reason a homotopy of maps splits into a homotopy on $A$ and one on $B$ so we get the same universal property in the homotopy category. The final thing we need to check is that this structure descends again to the derived category. This seems like it should be a general fact of (reasonable) localisations. Suppose we have a span $Z \rightarrow S \leftarrow A \times B$ splitting up the map to the product this is canonically equivalent to the pair $(Z \rightarrow S \leftarrow A),(Z \rightarrow S \leftarrow B)$, then given any pair of morphisms $(Z \rightarrow S_A \leftarrow A),(Z \rightarrow S_B \leftarrow B)$ we can complete the quasi iso square $S_A \leftarrow Z \rightarrow S_B$ to get some $S$ that acts as a roof on both of these spans, this is now of the form that we want to make a map to the product. This describes the isomorphism that is the universal property.

We can now start the second part of the question. That is that the canonical map $\mathbf{K}(\mathcal{A}) \to \mathbf{D}(\mathcal{A})$ is an additive functor, that is it's a preadditive functor, that is it's on each hom group a map of abelian groups. For this we just check that when we do the derived addition to $X = X \to Y, X = X \to Y$.  In this case we can choose common roof $X$ and hense our projection maps are just the identities and so the sum is just the sum in $\mathbf K (\mathcal A)$
\end{solution}

\end{document}
