\documentclass{article}

\usepackage{biblatex}
\addbibresource{bib.bib}

\usepackage[utf8]{inputenc}
\usepackage{amsmath}
\usepackage{amssymb}
\usepackage{amsthm}
\usepackage{amsfonts}
\usepackage{stmaryrd}
\usepackage{mathrsfs}
\newtheorem{theorem}{Theorem}[section]
\newtheorem{corollary}{Corollary}[theorem]
\newtheorem{lemma}[theorem]{Lemma}
\newtheorem{proposition}[theorem]{Proposition}
\theoremstyle{definition}
\newtheorem{definition}{Definition}[section]
\newtheorem{example}{Example}[section]
\newtheorem{exercise}{Exercise}[subsection]
\newtheorem{solution}{Solution}[exercise]
\usepackage{xcolor}
\usepackage{graphicx}
\usepackage{tikz}
\usepackage{tikz-cd}
\usepackage{tkz-euclide}
\usepackage{hyperref}
\usepackage{bm}
\usepackage{quiver} 
\usepackage{todonotes}
%\usepackage[a5paper, total={108mm, 160mm}]{geometry}

\pagecolor[HTML]{ECE1CB} %beige

\color[HTML]{756244} %brown 

\newtheorem{case}{Case}
\newcommand{\dd}[2]{\frac{\partial #1}{\partial #2}}
\newcommand{\oring}{\mathcal{O}}
\newcommand{\id}{\operatorname{id}}
\newcommand{\h}[1]{\widehat{#1}}
\newcommand{\height}{\operatorname{ht}}
\newcommand{\supp}{\operatorname{Supp}}
\newcommand{\im}{\operatorname{im}}
\newcommand{\coim}{\operatorname{coim}}

\newcommand{\ext}[1][1]{\operatorname{Ext}^{#1}}
\newcommand{\C}{\mathbb{C}}
\newcommand{\R}{\mathbb{R}}
\newcommand{\Q}{\mathbb{Q}}
\newcommand{\Z}{\mathbb{Z}}
\newcommand{\N}{\mathbb{N}}
\newcommand{\Pp}{\mathbb{P}}
\newcommand{\p}{\mathfrak{p}}
\newcommand{\q}{\mathfrak{q}}
\newcommand{\m}{\mathfrak{m}}
\newcommand{\del}{\partial}
\newcommand{\Gen}[1]{\langle #1 \rangle}
\newcommand{\spec}{\operatorname{spec}}
\newcommand{\Lim}[1]{\raisebox{0.5ex}{\scalebox{0.8}{$\displaystyle \lim_{#1}\;$}}}
\newcommand{\IN}{\operatorname{in}_{\prec}}
\newcommand{\ann}{\operatorname{ann}}
\newcommand{\Ass}{\operatorname{Ass}}
\newcommand{\coker}{\operatorname{coker}}
\newcommand{\pic}{\mathcal{P}ic}
\newcommand{\res}{\operatorname{res}}
\newcommand{\A}[2]{\mathbb{A}_#1^{#2}}
\newcommand{\Pj}[2]{\mathbb{P}_#1^{\#2}}
\newcommand{\cK}{\mathcal{K}}
\newcommand{\cM}{\mathcal{M}}
\newcommand{\cF}{\mathcal{F}}
\newcommand{\br}{\vspace{5mm}}

\newcommand{\V}{\mathbb{V}}
\newcommand{\I}{\mathbb{I}}

\newcommand{\cC}{\mathcal{C}}
\newcommand{\cD}{\mathcal{D}}
\newcommand{\cT}{\mathcal{T}}

\newcommand{\lil}{{\lambda \in \Lambda}}
\newcommand{\cL}{\mathcal{L}}
\newcommand{\cc}{\mathcal}
\newcommand{\rightrightrightarrows}{\substack{\rightarrow\\[-1em] \rightarrow \\[-1em] \rightarrow}}



\newcommand{\nul}{Nullstellensatz }
\newcommand{\newsect}[1]{\setcounter{subsection}{#1}\setcounter{exercise}{0}}

% \usepackage{ccfonts}
% \usepackage[T1]{fontenc}
% \usepackage{titlesec}

\usepackage[capitalise]{cleveref}

\title{Exercises on Derived Categories}
\author{Kyle Thompson}
\date{November 2025}

\begin{document}

\maketitle

I need to do exercises so I found this \href{https://homepages.math.uic.edu/~bshipley/krause.chicago_ex.pdf}{Exercises on derived categories, resolutions, and Brown
representability} to I suppose work through 

For the sake of consistancy denote
\begin{enumerate}
    \item $\mathcal{A}$ an abelian category
    \item $\mathbf{C}(\mathcal{A})$ the category of complexes of elements in $\mathcal{A}$
    \item $\mathbf{K}(\mathcal{A})$ the category of complexes modulo homotopy
    \item $\mathbf{D}(\mathcal{A})$ the derived category $\mathbf{K}(\mathcal{A})[\{\text{q-iso}\}^{-1}]$ with $\mathbf{D}^b, \mathbf{D}^+, \mathbf{D}^-$ being the subcategory of bounded, positive degree and negative degree complexes respectively
    \item For a localising system we can take the category of spans modulo some relations as the localisation, for this I use the convention from Gelfand-Manin, not that of the book this is from. That a morphism $X \to Y$ is a span $s: X \leftarrow Z \rightarrow Y:t$ with $s$ a quasi-iso and $t$ is just any morphism
\end{enumerate}
\pagebreak
\setcounter{section}{1} 
\newsect{2}
\begin{exercise}
Let $\mathcal{A}$ be an abelian category. Show that $\mathbf{K}(\mathcal A)$ and $\mathbf D (\mathcal A)$ are additive categories, and that the canonical functor $\mathbf{K} (\mathcal A) \to \mathbf D (\mathcal A)$ is additive
\end{exercise}
\begin{solution}
Clearly the hom sets for the category $\mathbf K (\mathcal A)$ admits an abelian group structure by just adding in each component in $\mathbf{C}(\mathcal{A})$ and then noting that \[H=\{f \vert f \simeq 0\}=\{f \vert \exists \phi, f = d\phi + \phi d\}\] is a subgroup as $d\phi + \phi d - (d\mu + \mu d)=d(\phi-\mu)+(\phi-\mu)d$ and $0 \in H$ so the structure decends naturally to the quotient. Since $\{q-iso\}$ form a localising class in $\mathbf K (\mathcal A)$ we can consider morphisms in $\mathbf{D} (\mathcal  A)$ as spans $X \rightarrow Z \leftarrow Y$ which have an abelian group structure since for two morphisms $(s \colon X \rightarrow Z \leftarrow Y \colon f )+ (s'\colon X \rightarrow  Z' \leftarrow Y \colon f')$ we can define their sum as follows, first we use the extensionality to construct some $W$ over both $Z, Z'$ so that the square 
\[\begin{tikzcd}
W \arrow[d,dashed, "a'"'] \arrow[r, dashed, "a"] & Z \arrow[d] \\
Z' \arrow[r]                      & X          
\end{tikzcd}\]
commutes and has all morphisms quasi iso, we can then construct the span 
\[\begin{tikzcd}
W \arrow[r, "f \circ a + f' \circ a'"] \arrow[d] & Y \\
X                                                &  
\end{tikzcd}\]
as the sum of our two morphisms. Since sadly this choice of $W$ isn't canonical, we must check that for any other choice of $W$ we get an equivalent span. This is surprisingly annoying. We can do so by taking some other common roof say $W'$ with projections onto the $Z$'s $b,b'$ we can find an object $W^\circ$ from the following extensions
\[\begin{tikzcd}
W^\Delta \arrow[d, dashed] \arrow[r, dashed] & W' \arrow[d] & W^\square \arrow[d, dashed] \arrow[r, dashed] & W' \arrow[d] & W^\circ \arrow[d, dashed] \arrow[r, dashed] & W^\square \arrow[d] \\
W \arrow[r]                                  & Z            & W \arrow[r]                                   & Z'           & W^\Delta \arrow[r]                          & X                  
\end{tikzcd}\]

This is almost what we want as each of the squares constructing $W^\Delta,W^\square$ tell us going via $W^\Delta$ we have the commutativity we want for $a,b$ and going via $W^\square$ we have commutativity for $a',b'$. Our issue is that we still lack canonical morphisms to $W,W'$ letting us do both at once.

As we have two natural routes, this can be fixed by noting that the maps $W^\circ \rightrightarrows W$ are equalised once we get to $X$ and so since $W \to X$ is a quasi iso they must be equalised from some quasi iso $V_* \to W^\circ$, doing the same for $W'$ we get two maps to $W^\circ$ so applying the extensionality one more time gives us some $V$ with quasi isomorphisms $\mu\colon V \to W, \mu' \colon V \to W'$, and since we've fixed our problem, $a\mu=b\mu', a'\mu=b'\mu'$. So we have a common roof for our two sums so they're equivalent and our construction is well defined.

We clearly have identity $id\colon X \rightarrow X \leftarrow Y \colon 0$ since for anything else $Z$ is a common roof so we just end up with $f+0=f:Z \to Y$ and $s:Z \to X$ so nothing chages. For any $(s \colon X \rightarrow Z \leftarrow Y \colon f )$ we have inverse $(s \colon X \rightarrow Z \leftarrow Y \colon -f )$ as in that case we can choose $W = Z$ leaving the triangle 
\[\begin{tikzcd}
Z \arrow[r, "0"] \arrow[d] & Y \\
X                          &  
\end{tikzcd}\]
Which is clearly equivalent to our identity element by the roof 
\[
\begin{tikzcd}
                                                 & Z \arrow[ld] \arrow[d, Rightarrow,no head,] \\
X \arrow[d, Rightarrow,no head,] \arrow[rd, "0"] & Z \arrow[d, "0"] \arrow[ld]                 \\
X                                                & Y                                          
\end{tikzcd}
\]
So we've shown that $\mathbf{K}(\mathcal{A}), \mathbf{D}(\mathcal{A})$ are preadditive, we now need to show that they have zero objects and products. We obviously have a zero objects in $\mathbf{C}(\mathcal{A})$ given by the zero complex, this then decends to a zero object in $\mathbf{K}(\mathcal{A})$ as the homotopy relation can only make the hom sets smaller, then in the derived category we have that the zero morphism is a quasi isomorphism from the zero complex to itself and hense this is a zero object in $\mathbf{D}(\mathcal{A})$ too.

For finite products we just need to check for binary products. We can find the product of two complexes by either doing $(A \oplus B)_i = A_i \oplus B_i$ or $(A \oplus B)_i = \bigoplus_{m+n=i}A_m \oplus B_n$ and honestly I'm not sure which. Ok there's a problem with the second one which is good for me. The problem is we would need potentially infinite products in $\mathcal A$ which may not exist. Instead we will try the first one, which honestly I much prefer to work with anyway. Given two complexes $A,B$ we can construct the product complex $A \oplus B$ having 
\[
(A \oplus B)^i = A^i \oplus B^i \quad d^i = \begin{pmatrix}d_A & 0 \\ 0 & d_b\end{pmatrix}
\]
We can see that this is a product in $\mathbf{C}(\mathcal{A})$ by zooming in to degree $0$ and noting that a map $Z \to A \oplus B$ we get the diagram
\[
\begin{tikzcd}
\vdots                                 & \vdots                   \\
Z^1 \arrow[u] \arrow[r, "{(f^1,g^1)}"] & A_1 \oplus B_1 \arrow[u] \\
Z^0 \arrow[u] \arrow[r, "{(f^0,g^0)}"] & A_0 \oplus B_0 \arrow[u] \\
\vdots \arrow[u]                       & \vdots \arrow[u]        
\end{tikzcd}
\]
Which clearly commutes iff $df=fd$ and $dg=gd$ ie that we have a map to $A$ and one to $B$, this is precicely the universal property of the product. This structure then descends to $\mathbf K (\mathcal{A})$ as for the same reason a homotopy of maps splits into a homotopy on $A$ and one on $B$ so we get the same universal property in the homotopy category. The final thing we need to check is that this structure descends again to the derived category. This seems like it should be a general fact of (reasonable) localisations. Suppose we have a span $Z \rightarrow S \leftarrow A \times B$ splitting up the map to the product this is canonically equivalent to the pair $(Z \rightarrow S \leftarrow A),(Z \rightarrow S \leftarrow B)$, then given any pair of morphisms $(Z \rightarrow S_A \leftarrow A),(Z \rightarrow S_B \leftarrow B)$ we can complete the quasi iso square $S_A \leftarrow Z \rightarrow S_B$ to get some $S$ that acts as a roof on both of these spans, this is now of the form that we want to make a map to the product. This describes the isomorphism that is the universal property.

We can now start the second part of the question. That is that the canonical map $\mathbf{K}(\mathcal{A}) \to \mathbf{D}(\mathcal{A})$ is an additive functor, that is it's a preadditive functor, that is it's on each hom group a map of abelian groups. For this we just check that when we do the derived addition to $X = X \to Y, X = X \to Y$.  In this case we can choose common roof $X$ and hense our projection maps are just the identities and so the sum is just the sum in $\mathbf K (\mathcal A)$
\end{solution}
\pagebreak
\newsect{4}
\begin{exercise}
    Let $\mathcal{A}$ be an abelian category and denote by $T$ the class of all quasi-isomorphisms in $\mathbf{C}(\mathcal{A})$. Show that two maps $f,g \colon X \to Y$ in $\mathbf{C}(\mathcal{A})$ are identified by the canonical functor $\mathbf{C}(\mathcal{A})\to \mathbf{C}(\mathcal{A})[T^{-1}]$ if $f - g$ is null homotopic 
\end{exercise}
\begin{solution}
    Note we have a problem, it's not the case that $T$ is a localising system in $\mathbf{C}(\mathcal{A})$ and so we don't have an easy way to attack the problem.

    We start the only way that we can, suppose that $d \psi + \psi d = g - f$ we want to squeeze out a quasi isomorphism somewhere here. [At this point I read the first sentence of the proof in Gelfand Manin which is just] we write $g=f+d\psi+\psi d$. And then you construct the cone of the two morphisms. Since cones are unique up to (not neccacarily unique) isomorphism in a triangulated category, if their cones are quasi iso, that pushes forward to an iso of distinguished triangles and hense the two morphisms are the same. Recalling that the cone of $f$ has differential
    \[
\begin{pmatrix}d_{X[1]} & 0 \\ f[1] & d_Y\end{pmatrix}
    \]
We need a map $\begin{pmatrix}A & B \\ C & D \end{pmatrix}$ so that 
    \[
\begin{pmatrix} d_{X[1]} & 0 \\ f[1] & d_Y\end{pmatrix} \begin{pmatrix}A & B \\ C & D \end{pmatrix} = \begin{pmatrix}A[1] & B[1] \\ C[1] & D[1] \end{pmatrix}\begin{pmatrix}d_{X[1]} & 0 \\ g[1] & d_Y\end{pmatrix} 
    \]
    That is 
    \begin{align*}
        d_{X[1]}A &= A[1]d_{X[1]} + B[1]g[1] \\
        f[1]A+d_Y C &= C[1]ds_{X[1]} + D[1]g[1] \\
        d_{X[1]}B &= B[1]d_{Y} \\
        f[1]B + d_Y D &= D[1]d_Y
    \end{align*}
    The second equation suggests we let $C = \psi^i$, $A =id, D = id, B = 0$, blindly pushing this through in the $i$'th component we get
    \begin{align*}
        d_{X[1]}&= d_{X[1]} + 0 \\
        f^{i+1} +d_Y \psi^i &= \psi^{i+1}d_{X[1]} + g^{i+1} \\
        0 &= 0 \\
        0 + d_Y  &= d_Y
    \end{align*}
Which are all true, the second one is true thanks to $\psi$ being a homotopy and $d_{X[1]}^i=-d_X^{i+1}$. This matrix 
\[
\begin{pmatrix}
    id & 0\\
    \psi & id
\end{pmatrix}
\]
Is a quasi isomorphism since litterally look at it. Once we pass to homology we see that each map has determinant $1$ and so is invertable. This gives us the quasi-isomorphism of cones and hense of distinguished triangles and so in $\mathbf{C}(\mathcal{A})[T^{-1}]$, $f=g$

Ok so reading a full solution I jumped the gun a little there, we dont know that $\mathbf{C}(\mathcal{A})[T^{-1}]$ is triangulated lmao. Instead you use this quasi isomorphism to make an iso of Cylinders
\[
\operatorname{Cyl}(f) = X \oplus X[1] \oplus Y \quad d_{\operatorname{Cyl}(f)}^i(x_1, x_2, y) = (d_X x - x_2, - d_X x_2, f(x_2)+d_Y y)
\]
Thanks to the diagram with exact rows
\[
\begin{tikzcd}
            & 0 \arrow[r]                                & Y \arrow[r] \arrow[d, "\alpha"]                    & \operatorname{Cone}(f) \arrow[r] \arrow[d, Rightarrow, no head] & {X[1]} \arrow[r] & 0 \\
0 \arrow[r] & X \arrow[r] \arrow[d, Rightarrow, no head] & \operatorname{Cyl}(f) \arrow[r] \arrow[d, "\beta"] & \operatorname{Cone}(f) \arrow[r]                                & 0                &   \\
            & X \arrow[r]                                & Y                                                  &                                                                 &                  &  
\end{tikzcd}
\]
With $\alpha$,$\beta$ quasi iso, $\beta \alpha = id$, $\alpha \beta \simeq id$. Calling the cylinder map $C$ we can glue the two diagrams together to conclude that $f = \beta_f\bar{f}, g=\beta_g C \bar{f}$ so when we localise $\beta_f^{-1}=\alpha_f$ so $g = \beta_g C \alpha_f f = f$ by commutativity.
\end{solution}
\pagebreak
\newsect{5}
\begin{exercise}
    Let $\mathcal{A}$ be the module category for some ring $\Lambda$. Show that $\hom_{\mathbf{D}(\mathcal{A})}(\Lambda,X) \cong H^0(X)$ for every complex of $\Lambda$ modules
\end{exercise} 
\begin{solution}
    First note that because homology groups respect homotopy we can equivocate homotopy representatives and homotopy classes relatively freely. Suppose then that we have a morphism $\Lambda \to X$, that is a $H^0$ complex $Z$ with q-iso to $\Lambda$ and a map to $X$, taking homology of this triangle then we get $\Lambda \leftarrow H^0(Z) \rightarrow H^0(X)$ where the first map is a quasi isomorphism, so we get a canonical element of $H^0(X)$ by pulling back $1$ and then pushing it down to $H^0(X)$. This gives us a group homomorphism $\hom_{\mathbf{D}(\mathcal{A})}(\Lambda, X) \to H^0(X)$, it's well defnied as if we had another roof it's the same up to coherent quasi isomorphism so gives the same maps in homology. Additionally it's a group hom as if we add together two maps we have seen that this just adds together the maps modulo quasi isomorphisms that vanish when we take $H^0$ and so we end up with $(f+g)(1)=f(1)+g(1)$ so we do in fact get a group homomorphism. It suffices now to check that it's an isomorphism. Clearly it's surjective by taking $\Lambda = \Lambda \to X$ sending $1$ to any element. Then if $f \mapsto 0$ we have that the induced map on homology $H^0(\Lambda)=\Lambda\to H^0(X)$ maps $1$ to $0$ and is thus the zero map as it must respect the $\Lambda$-action. This is probably not enough, note the follwoing lemma [Exercise III.4.1] in Gelfand-Manin
    \begin{lemma}
        For $f \in \hom_{\mathbf{K}(\mathcal{A})}(X,Y)$
        TFAE \begin{enumerate}
            \item $f = 0$ in $\mathbf{D}(\mathcal{A})$
            \item There is a quasi isomorphism $s\colon Y \to Z$ so that $sf \simeq 0$
            \item There is a quasi isomorphism $t \colon W \to X$ so that $ft \simeq 0$
        \end{enumerate}
    \end{lemma}
    So in general we don't expect that inducing the zero map in homology is enough to get the map is zero. It is however neccacary so is a healthy sign toward $f=0$. By freeness of $\Lambda$ as a $\Lambda$ module we can find a map $\Lambda \to Z^0$ so that $\Lambda \to Z^0 \to \Lambda$ is the identity map. Hense we can replace $Z$ with $\Lambda$ as it acts as a roof over $Z$ so we have $\Lambda = \Lambda \to X$ where since the map $\Lambda \to X$ induces the zero map in homology it must be homotopic to the zero map since we end up in the image we can find a factorisation $\Lambda \to X^{-1} \to X^0 = \Lambda \to X^0$ which gives us the homotopy we need as everything else is zero. Hense $f=0$ and we're done
\end{solution}
\pagebreak
\begin{exercise}
    Let $\mathcal{A}$ be an abelian category. Show that the canonical functor $\mathcal{A} \to \mathbf{D}(\mathcal{A})$ identifies $\mathcal{A}$ with the full subcategory of complexes $X$ in $\mathbf{D}(\mathcal{A})$ that have $H^n X=0 \ \forall n \neq 0$ 
\end{exercise}
\begin{solution}
    We have to show two things, that this is fully faithful, and that it's essentially surjective. To first do essential surjection we note that $X\cong H^0(X)$ in the derived category as we have quasi isomorphisms $H^0(X) \leftarrow V_X \rightarrow X$ where $V_X$ is the truncation $\dots \rightarrow X^{-1} \to \ker(d^0) \to 0$ and so these compose to an isomorphism in the derived category.

    To show that this restriced map is full and faithful we can take $X,Y \in \mathcal{A}$ and any map $X \leftarrow Z \rightarrow Y$ we can take $H^0$ to get maps $X=H^0(X) \leftarrow H^0(Z) \rightarrow H^0(Y)=Y$ which will then fit into the equivalence 
    \[
\begin{tikzcd}
                        & V_Z \arrow[ld] \arrow[rd] &                              \\
Z \arrow[d] \arrow[rrd] &                           & H^0(Z) \arrow[d] \arrow[lld] \\
X                       &                           & Y                           
\end{tikzcd}
    \]
    So we may assume that the roof too is in $\mathcal{A}$ so the quasi isomorphism is just a true isomorphism and so the maps are precicely the maps $X \to Y$ so we are fully faithful
\end{solution}
\pagebreak
\newsect{6}
\begin{exercise}
    Let $\mathcal{A}$ be the category of vector spaces over a field $k$. Describe all objects and morphisms in $\mathbf{D}(\mathcal{A})$
\end{exercise}
\begin{solution}
    By freeness of vector spaces we know that since $H^i(X)$ is a vector space we can construct maps from it quite easily, in fact for any $d$ we have the quotient map $\ker d \to H^i(X)$ and hense can construct a section $H^i(X) \to \ker d$, which therefor induces a map $H^i(X) \to X^i$. This motivates the following lemma
    \begin{lemma}
        For $X \in \mathbf{K}(k-\mathbf{vect})$ there exists a quasi isomorphism $H^\bullet(X) \to X$ where $H^\bullet(X)$ denotes the complex
        \[
\begin{tikzcd}
\dots \arrow[r, "0"] & H^{-1}(X) \arrow[r, "0"] & H^0(X) \arrow[r, "0"] & H^1(X) \arrow[r, "0"] & \dots
\end{tikzcd}
        \]
    \end{lemma}
    \begin{proof}
        First note that the homology of this complex at $i$ is $\ker 0 / \im 0 = H^i(X)$ and so a quasi isomorphism is reasonable to expect.
        
        We construct the map as described before, ie we lift through the kernel to make a map $H^i(X) \to X^i$. This map is clearly a quasi isomorphism as it's a chain map since we land in the kernel so $fd = df = 0$ and is quasi iso since the induced map on kernels induces an isomorphism in the quotients by construction
    \end{proof}
    This then gives us a precise description of the category
    \begin{proposition}
        There is an equivalence of categories $\prod_\Z \mathcal{A} \to \mathbf{D}(\mathcal{A})$ sending a $\Z$ indexed tuple of vector spaces $(\dots,X^{-1},X^{0},X^{1},\dots)$ to the sequence \[\dots \to X^{-1}\to X^0\to X^1\to\dots
        \]
        With $d=0$
    \end{proposition}
    \begin{proof}
        We showed in the previous lemma that this map is essentially surjective, it just then suffices to show that it's full and faithful. Suppose we have two complexes with zero differential $X,Y$ and a map between them $X \leftarrow Z \rightarrow Y$ we know from the previous lemma that there's a quasi isomorphism $X \to Z$ that maps down to the identity so we can assume our span is of the form $X = X \rightarrow Y$ Thus it's clear that the functor is full since for any map $X \to Y$ it is a sequence of maps in each component and hense id just the image of some map in the product category. If we then assume that two morphisms end up equivalent, that is we have a common roof  
        \[
\begin{tikzcd}
                                             & Z \arrow[ld] \arrow[rd] &                                              \\
X \arrow[d, Rightarrow, no head] \arrow[rrd] &                         & X \arrow[lld, Rightarrow, no head] \arrow[d] \\
X                                            &                         & Y                                           
\end{tikzcd}
        \]
        We see that the two maps $Z \to X$ are the same and hense the maps $X \to Y$ are the same and so are homotopic. There are no interesting homotopies between these two complexes as $d\psi +\psi d = 0$ so two maps are homotopic if and only if they are equal pointwise so the two maps must've been the same in the product
    \end{proof}
\end{solution}
\pagebreak
\begin{exercise}
    Let $\mathcal{A}$ be the category of finitely generated abelian groups and $\mathcal{P}$ be the category of fintely generated free abelian groups. Describe all objects and morphisms in $\mathbf{D}^b(\mathcal{A})$. Show that the canonical functor $\mathbf{K}^b(\mathcal{P}) \to \mathbf{D}^b(\mathcal{A})$ is an equivalence
\end{exercise}
\begin{solution}
    While this is stated as two parts, I feel like an eqivalence of categories is exactly what we want when describing all objects and morphisms. 

    We want to do some resolution-y thing since this is exactly what we do when deriving functors\footnote{at this point i want to google eilenburn something resolution 
    (names aren't my strong suit) but i will abstain for now}, we find a class of objects so that this equivalence exists and use the equivalence to derive our functor. Finitely generated abelian groups are nice since they're precicely quotients of free groups and hense for any $G$ we have free resolution
    \[
0 \to \Z^m \to \Z^n \to G \to 0
    \]
    Note that any map $G \to H$ lifts to a map of these resolutions (unique up to homotopy) as free objects are projective, the lifts can here be constructed in the obvious way so we get a big ol diagram
    \[
\begin{tikzcd}
\dots \arrow[r]   & 0 \arrow[d] \arrow[r]           & 0 \arrow[d] \arrow[r]        & 0 \arrow[d] \arrow[r]        & \dots   \\
\dots \arrow[r]   & \Z^{m_{-1}} \arrow[r] \arrow[d] & \Z^{m_0} \arrow[r] \arrow[d] & \Z^{m_1} \arrow[d] \arrow[r] & \dots   \\
\dots \arrow[r]   & \Z^{n_{-1}} \arrow[r] \arrow[d] & \Z^{n_0} \arrow[r] \arrow[d] & \Z^{n_1} \arrow[d] \arrow[r] & \dots   \\
\dots \arrow[r] & X^{-1} \arrow[r] \arrow[d]      & X^0 \arrow[r] \arrow[d]      & X^1 \arrow[r] \arrow[d]      & \dots \\
                & 0                               & 0                            & 0                            &      
\end{tikzcd}
    \]
    Which we somehow want to extract a complex of free groups that's quasi isomorphic to our original complex. If we started with just $X$ concentrated in a single degree our free resolution would give the following quasi iso
    \[
\begin{tikzcd}
0 \arrow[r] & \Z^m \arrow[r] & \Z^n \arrow[r] \arrow[d] & 0 \\
            & 0 \arrow[r]    & G \arrow[r]              & 0
\end{tikzcd}
    \]
    But our problem is with more groups that there's overlaps, if we suggestively reshape our diagram as follows 
    \[\begin{tikzcd}
	\dots & 0 & 0 & 0 & \dots \\
	& \dots & {\Z^{m_{-1}}} & {\Z^{m_0}} & {\Z^{m_1}} & \dots \\
	&& \dots & {\Z^{n_{-1}}} & {\Z^{n_0}} & {\Z^{n_1}} & \dots \\
	&& \dots & {X^{-1}} & {X^0} & {X^1} & \dots \\
	&&& 0 & 0 & 0
	\arrow[from=1-1, to=1-2]
	\arrow[from=1-2, to=1-3]
	\arrow[from=1-2, to=2-3]
	\arrow[from=1-3, to=1-4]
	\arrow[from=1-3, to=2-4]
	\arrow[from=1-4, to=1-5]
	\arrow[from=1-4, to=2-5]
	\arrow[from=2-2, to=2-3]
	\arrow[from=2-3, to=2-4]
	\arrow[from=2-3, to=3-4]
	\arrow[from=2-4, to=2-5]
	\arrow[from=2-4, to=3-5]
	\arrow[from=2-5, to=2-6]
	\arrow[from=2-5, to=3-6]
	\arrow[from=3-3, to=3-4]
	\arrow[from=3-4, to=3-5]
	\arrow[from=3-4, to=4-4]
	\arrow[from=3-5, to=3-6]
	\arrow[from=3-5, to=4-5]
	\arrow[from=3-6, to=3-7]
	\arrow[from=3-6, to=4-6]
	\arrow[from=4-3, to=4-4]
	\arrow[from=4-4, to=4-5]
	\arrow[from=4-4, to=5-4]
	\arrow[from=4-5, to=4-6]
	\arrow[from=4-5, to=5-5]
	\arrow[from=4-6, to=4-7]
	\arrow[from=4-6, to=5-6]
\end{tikzcd}\]
We could consider taking the sum in each component ie let $F^i = \Z^{n_i} \oplus \Z^{m_{i+1}}$ with differential $d_F \colon (t,s) \mapsto (fs-dt, ds)$ where $d$ is the lift of the differential  by starting from one end and moving along we can choose a lift so that $d^2 = 0$ and hense this differential also satisfies $d_F^2=0$ as
\[
d_Fd_F(t,s) = d_F(fs-dt,ds) = (f(ds) - d(fs-dt), dds) = (fds - dfs + ddt, dds) = 0
\]
Note that this is precicely the mapping cone for $Z^{m_i} \to Z^{n_i}$ and hense has homology fitting into the same long exact sequence as the one $X$ fits into. As the projection map commutes everything we know that by the five lemma it induces an isomorphism in homology so is quasi iso. This shows that the map $\mathbf{K}^b(\mathcal{P}) \to \mathbf{D}^b(\mathcal{A})$ is essentially surjective.

Checking fully faithfullness is then not too bad, for a map $X \leftarrow Z \rightarrow Y$ since $X$ is free we can find a section $X \to Z$ and hense we can replace $Z$ with the roof $X$ so everything is of the form $X = X \to Y$ and so the functor is full. It's then faithful as if we have two equivalent morphisms we draw the same diagram as in the previous exercise and conclude that the maps are homotopic. This shows that our map $\mathbf{K}^b(\mathcal{A}) \to \mathbf{D}^b(\mathcal{A})$ is an equivalence.
\end{solution}
\pagebreak
\begin{exercise}
    Let $k$ be a field and consider the following finite dimensional algebras
    \[
    \Lambda_1 = \begin{pmatrix}
        k & k & k \\ 0 & k & k \\ 0 & 0 & k
    \end{pmatrix}\quad
    \Lambda_2 = \begin{pmatrix}
        k & k & 0 \\ 0 & k & 0 \\ 0 & k & k
    \end{pmatrix}\quad
    \Lambda_3 = \Lambda_1 / I, \ I = \begin{pmatrix}
        0 & 0 & k \\ 0 & 0 & 0 \\ 0 & 0 & 0
    \end{pmatrix}
    \]
    Describe, in each case, the category $\mathcal{A}_i$ of finite dimensional $\Lambda_i$-modules and its derived category $\mathbf{D}^b(\mathcal{A}_i)$. Here are some hints
    \begin{enumerate}
        \item $\mathcal{A}_1$ and $\mathcal{A}_2$ are hereditary categories, but $\mathcal{A}_3$ is not.
        \item Each object in $\mathcal{A}_i$ or $\mathbf{D}^b(\mathcal{A}_i)$ decomposes essentially uniquely into a finite number of indecomposable objects
        \item The indecomposable projective $\Lambda_i$ modules are $E_{jj}\Lambda_i, j = 1,2,3$
        \item $\Lambda_1$ and $\Lambda_2$ have each $6$ pairwise non-isomorphic indecomposable modules, and $\Lambda_3$ has $5$
        \item $\ext[n]_{\Lambda_i}(X,Y)$ has $k$ dimenion at most $1$ for all indecomposable $\Lambda_i$-modules $X,Y$ and $n \geq 0$
    \end{enumerate}
    The \textit{Auslander-Reiten quiver} provides a convenient method to display the categories $\mathcal{A}_i$ and $\mathbf{D}^i(\mathcal{A}_i)$, because the morphism spaces between indecomposable objects are at most one dimensional. This quiver is defined as follows. The vertices correspond to the indecomposable objects. Put an arrow $X \to Y$ between two indecomposable objects if there is an irreducible map $\phi: X \to Y$ (where $\phi$ is irreducible if $\phi$ is not invertable and any factorisation $\phi = \phi'' \circ \phi'$ implies that $\phi'$ is a split monomorphism or $\phi''$ is a split epimorphism)
\end{exercise}
\begin{solution}
    \todo{this looks involved and im ill}
\end{solution}
\newsect{7}
\pagebreak
\begin{exercise}
    Let $\mathcal{A}$ be an abelian category. Show that the canonical functor $\mathbf{D}^b(\mathcal{A})\to \mathbf{D}(\mathcal{A})$ is fully faithful
\end{exercise}
\begin{solution}
    This is equivalent to showing that for any span $X\leftarrow Z\rightarrow Y$ for $X,Y$ bounded complexes, we can assume that $Z$ is bounded. Since $Z$ has bounded cohomology we can assume that it's bounded above by taking the quasi isomorphism $V_{Z[i]}[-i] \to Z$ for some appropriate choice of $i$. We can also cut off the negative part by taking the cokernel truncation of this $V\rightarrow U^{V[j]}[-j]$, far enough along that all of the maps $Z\rightarrow X,Z\rightarrow Y$ are the zero map. This allows us to factor the maps to $X,Y$ through this quasi isomorphism and since we cut off both ends this is an equivalent roof that's totally bounded so we are done.
\end{solution}
\pagebreak
\begin{exercise}
    Let $\mathcal{A}$ be an abelian category and denote by $\mathcal{I}$ the full subcategory of injective objects. Suppose that $\mathcal{A}$ has enough injective objects. Then the canonical functor $\mathbf{K}^+(\mathcal{I}) \to \mathbf{D}^+(\mathcal{A})$ is an equivalence.
\end{exercise}
\begin{solution}
    We need to know $2$ things, every complex is quasi-isomorpic to a complex of injective objects, and that a map of injective complexes is the same when localised at quasi-isomorphisms

    The first thing is done by recalling from before what we did with projective objects before. 
    \[
\begin{tikzcd}
\dots \arrow[r] & X^{-1} \arrow[d] \arrow[r]                                & X^0 \arrow[d] \arrow[r]                         & X^1 \arrow[d] \arrow[r]                              & \dots                      &                 &       \\
\dots \arrow[r] & I^0_{-1} \arrow[rd, "\del_{-1}^0"'] \arrow[r, "d^0_{-1}"] & I_0^0 \arrow[rd, "\del_0^0"] \arrow[r, "d^0_0"] & I_1^0 \arrow[rd, "\del_{1}^0"] \arrow[r, "d^0_{-1}"] & \dots                      &                 &       \\
                & \dots \arrow[r]                                           & I_{-1}^1 \arrow[r] \arrow[rd]                   & I_0^1 \arrow[rd] \arrow[r]                           & I^1_1 \arrow[rd] \arrow[r] & \dots           &       \\
                &                                                           & \dots \arrow[r]                                 & I_{-1}^0 \arrow[r]                                   & I^2_0 \arrow[r]            & I^2_1 \arrow[r] & \dots
\end{tikzcd}
    \]
    
    Since $\mathcal{A}$ has enough injectives for each object in a complex in $\mathbf{K}(\mathcal{A})$ we have an injective resolution $A \to I^0 \to I^1 \to I^2 \dots$ we can then by standard results for injective objects descend the differential to a differential on each $I_i^0 \to I_{i+1}^0$ by starting from the bottom to ensure each lift satisfies $d^2=0$ we then define a familiar complex
    \[
J^n = \bigoplus_{i+j=n} I_{i}^{j}
    \]
    With differential $(-1)^n(d-\del)$, this is a differential as its double application is just difference of squares, but as both square to zero it's the zero map. This is then a quasi isormohpism as the kernel is where $d$ and $\del$ agree, which is precicely when they come from something in the image by exactness of $\del$ + commutativity. That is unless we're at the end of the chain where there is no image, so the kernel is just the kernel of the $d_X$, the image has the same property so as the map is just inclusion of the kernel we conclude that this is a quasi isomorphims.

    By standard results in homological algebra each of these injective resolutions are unique up to homotopy and hense this total complex is a unique resolution up to homotopy, this gives us our fully faithful as $X \leftarrow Z \rightarrow Y$ $X$ is an injective resolution of $Z$ and hense any map $Z \to Y$ is uniquely determined as a map $X \to Y$ since $Y$ is injective so this is fully faithful \todo{rewrite this too ill}
\end{solution}
\end{document}
